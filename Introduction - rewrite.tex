%Wildlife montoring important (declines in populations, global declines)
Animal population size is one of the fundamental measures needed in ecology and conservation. The absolute size of a population has important implications for a range of issues such as genetic diversity \citep{o1985genetic, fischer2000genetic, willi2005threefold}, sensitivity to stochastic fluctuations \citep{richter1972extinction,wright1983stochastic} and risk of extinction \citep{purvis2000predicting}.

%Wildlife monitoring tech growing in sophication and widespread (remote sensors visual and acoustic)
Sensor technology, such as camera traps \citep{rowcliffe2008surveys,ahumada2011community} and acoustic detectors \citep{ofarrel1999comparison,mellinger2007fixed,jones2011indicator} are becoming increasingly used to survey animal populations \citep{rowcliffe2008surveys, kessel2014review}, as they are efficient, relativity cheap and non-invasive \citep{gese2001monitoring, o2003crouching, silveira2003camera}, allowing for surveys over large areas and long periods. 


%Difficult to estimate abundance and densities (needed for monitoring)
However, the problem of converting sampled count data to estimates of density remains. 


%Current methods are often inadequate because of specific data requirements (marked indivudals)
The preferred method for estimating density if individuals can be recognised is capture-recapture e.g. \citep{karanth1995, trolle2003estimation, soisalo2006estimating, trolle2007camera}. If individual recognition is impossible but the distance between animal and sensor can be estimated, distance sampling methods can be used to estimate density, although these often ignore animal movement which may bias estimates \citep{barlow2005estimates, marques2011estimating}

%REM developed for camera trap data, doesnt need this assumptions 
The gas model has been modified for use with camera traps, which have an angle of detection (the angle within which an animal can be detected)  up to $\pi$ radians \citep{rowcliffe2008estimating}, removing the need for individual recognition and distance estimation. 

%but has limitations
This angle limitation may not apply to acoustic detectors \citep{adams2012you}, or all types of camera traps. Furthermore, animal calls may be directional, and the existing random encounter model (REM) designed for camera trap data does not handle this case.

%Specifically sensor widths - different environments might need more flexibility (examples) and signal directionality assumptions (examples) - method has been optimised for terrestrial large animals
Fisheye lenses have been developed for monitoring wildlife and the canopy, but any count data collected with these cameras could not be inputted into the REM. 
Any camera looking down on a terrestrial landscape or in a marine environment would have a circular view point, and once again this data would not be suitable for use with the REM. 

%acoustic monitoring becoming more common method of monitoring but has additional sensor width problems (examples) and directionality of signals (examples). 
There has been a sharp rise in the number of publications with passive acoustic detectors, with fewer than 10 studies published in 2000, and more than 60 published in 2010, a more than 6 fold increase in a decade \citep{kessel2014review}. 
Different families and species of animals will display different directionality in their calls, there will even be different widths of call from the same animal depending on that calls purpose \citep{jakobsen2012convergent}

%Currently the count data from acoustic monitoring is used for monitoring in different ways (examples). 
Acoustic monitoring is being developed for study the interactions of animals and their environments \citep{blumstein2011acoustic, straight2014passive, marcoux2011local, rogers2013density}, as well as the presence and relative abundances of species \citep{mckown2012wireless, marcoux2011local}, and biodiversity of an area \citep{ depraetere2012monitoring}

%Another limitation to using these sensors is lack of guidelines on how survey effort impacts the accuracy and precision of estimates (guidelines would be useful. practically can only be done within simulated environment, rarely done)
Simulation of animal movement to evaluate monitoring methodology technology has been completed before \citep{ivan2013using, rees2011testing} in order to access the 

%We develop gREM and tested it and come up with specific recommedations for its use (providing code for implementation). we controlled for different animal movement by testing the robustness of gREM with different movement model (examples).
In this study we create a generalised REM (gREM), as an extension to the camera trap model of \citep{rowcliffe2008estimating}, to estimate absolute density from count data from acoustic detectors, or camera traps, where the sensor width can vary from 0 to $2\pi$ radians, and the acoustic signal given off from the animal can be directional (we call the width of an animals acoustic signal a signal width). 
We provide an R \citep{R} script for the implementation of the gREM in Appendix S1.  
 We tested the model using simulations in order to assess the validity of the models and in order to give suggestions for best practice. Specifically, we test that the analytical model can accurately predict density when the assumptions of a homogeneous environment and straight-line animal movement are met. We went on to test the accuracy of the model if the assumptions about animal movement were broken, including stop-start movement and correlated and random walks.
